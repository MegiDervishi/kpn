\documentclass{article}
\usepackage{hyperref}
\usepackage{xcolor}
\usepackage{enumitem}
\usepackage[english]{babel}
\usepackage{graphicx}
\usepackage{lmodern}
\usepackage{kpfonts}
\usepackage{subcaption}
\usepackage{enumitem}
\usepackage{wrapfig}
\title{Kahn}
\author{Nizar Ghandri, Megi Dervishi, Christophe Saad}
\date{\today}

\begin{document}


\maketitle

\section{Sequential}
The implentation of the sequencial part is manly taken from the paper  A poor man’s concurrency monad. We consider the process as a monad transformer which takes an action (which contains the actual computation) and link it a continuation. Its type is $$\texttt{type 'a process = ('a -> action) -> action}$$
The type action is $$\texttt{type action  = Atom of (unit -> action) | Fork of action * action | Stop}$$
We use \texttt{Fork} to instanciate new processes and \texttt{Atom} to represent a computation.\\
In the \texttt{run} function, we recreate a pipeline containing the first \texttt{Fork} action to execute. Each time a \texttt{Fork} is executed, we push the two actions in the pipeline. When an \texttt{Atom} is read, we execute its computation and store his continuation back in the pipeline. This procedures ends when all the continuations are \texttt{Stop}.

\section{Mandelbrot}
This application consits of the computation and visualization of the Mandelbrot set using the implemented kahn network.
\subsection{Madelbrot set}
The Mandelbrot set $\mathcal{M}$ is the set of complex numbers $c$ such that the point $z = 0$ of the polynomial $P_c(z) = z^2 + c$ has an orbit which remains bounded. \\
In other words, this set can be seen as the set of complex numbers $c$ for which the application $z_{n+1} = z_n + c$ starting from $0$ for $n \rightarrow \infty$ remains bounded.

\subsection{Properties}
A very useful property of the mandelbrot set which will make its computation easier is the following:
	$$c \in \mathcal{M}  \quad \Longleftrightarrow \quad |\mathcal{P}^n_c(0)| \leq 2 \quad \forall n\geq 0$$
The sequence is certainly divergent if the modulus of z crosses 2.

\subsection{Implementation}
We give here a quick description of our implementation.

\subsubsection{Objective}
Our objective is to compute and plot the Mandelbrot set $\mathcal{M}$ on a given window and range of complex numbers.

\subsubsection{Evaluation of points}
We map each pixel of the graph to a complex value depending on a given origin and a zoom scale (radius around the origin).\\
A common method for evaluating a point and checking whether it belongs to $\mathcal{M}$ consists of iterating over the sequence described above starting from $0$. The algorithm stops when the maximum number of steps (given as input) is reached or if we find that $|z| > 2$ (from previous property). An obvious optimization technique would be testing $|z|^2 > 4$ in order to avoid the squared root operation of the modulus.

\subsubsection{Colors}
A naive implementation would be to use two colors, one for pixels which belongs to $\mathcal{M}$ and another for those that don't. Another implementation technique to visualize similarity in points would be to use the number of iterations needed for the evaluation loop to exit. We can then visualize the deep points (those which needed the maximum number of steps) and the border points.\\
In our algorithm, we use a technique to visualize the equipotential lines of $\mathcal{M}$.\\
We replace the threshold $2$ by another value $r$ which can be seen as the escape radius. We use as inputs $r$ and the number of iterations $n$ to compute the potential $$V(c) = \frac{\log (\max (1, |z_n|))}{2^n}$$
This technique stays coherent in the sense that $V(c) = 0$ if and only if $c \in \mathcal{M}$. More information about the potential and required adjustments \href{https://www.math.univ-toulouse.fr/~cheritat/wiki-draw/index.php/Mandelbrot_set#The_potential}{\color{blue}here}. 


\subsubsection{Kahn processes}
We divide the output window into sections, each one will be handled by a process. We use one last process for plotting the resulting values of the processes computations.\\
Computing processes send points values (coordinates and color assigned) to the channel where they wait to be retrieved by the plotting process.

\subsection{Usage}
\subsubsection{Running mandelbrot}
The mandelbrot application can be ran with arguments (\texttt{arg default type})
\begin{itemize}[label={}]
	\item \texttt{-w 1300 int} Width of the window
	\item \texttt{-h 1000 int} Height of the window
	\item \texttt{-n 1000 int} Number of iterations
	\item \texttt{-p 1 int} Number of processes for computation (must divide width)
	\item \texttt{-xo -0.5 float} Real part of origin
	\item \texttt{-yo 0. float} Imaginary part of origin
	\item \texttt{-z 1. float} Zoom value (radius around origin)
	\item \texttt{-r 4. float} Escape radius value 
	
\end{itemize}

\subsubsection{Some views}
\begin{itemize}[label={}]
	\item \texttt{-xo -0.7463 -yo 0.1102 -z 0.005}
	\item \texttt{-xo -0.7453 -yo 0.1127 -z 0.00065}
	\item \texttt{-xo -0.16 -yo 1.0405 -z 0.026}
	\item \texttt{-xo -0.925 -yo 0.266 -z 0.032}
	\item \texttt{-xo -0.748 -yo 0.1 -z 0.0014}
	\item \texttt{-xo -0.722 -yo 0.246 -z 0.019}
	\item \texttt{-xo -0.235125 -yo 0.827215 -z 0.00004}
	\item \texttt{-xo -0.81153120295763 -yo 0.20142958206181 -z 0.0003}
\end{itemize}\section{Network}
This implementation of Khan aims at distributing the processes in multiple computers in parallel. It makes use of the module \texttt{Unix} for creating a classic TCP socket communication with an IPv4 or IPv6 address; makes use of the module \texttt{Marshal} to convert the value into bytes in order to be sent/received in the channel. The \texttt{put value output} function makes use of the unix function \texttt{Unix.send } to send the particular value through the socket \texttt{ output} and the \texttt{get input} function makes use of \texttt{Unix.receiv} to receive a particular value. The \texttt{doco} function takes a list of processes and runs them in parallel through the \texttt{Thread} module.

\section{Tic Tac Toe game}

This application makes use of the above Khan implementation. It aims at creating a classic Tic Tac Toe game between two players. The file is \texttt{tictactoe.ml}.

\subsection{Design of the communication }

In a classic client/server architecture, the server would have to decide the execution and communication of the khan processes between the two clients. However this implementation is a bit of over kill on the server side. In our implementation the communication happens directly between the clients where one of them becomes the host as in the following scheme:

%include graph her Client (server_hos) <=> Other Client 

In this implementation we use one terminal and one graphics window. It may seem that since there are not two communicating terminals/graphic windows the connection between the sockets has not been established, but in fact the "client host" and the "client" are two khan processes that are executed in parallel and communicate with each other only through khan channels. An improvement of this can be to establish the communication in two terminals and two graphic windows. To achieve the second is best to use a different module beside \texttt{Graphics}. 

\subsubsection{Structure of messages}
The messages that are sent through the channel fall into four categories (\texttt{MYM}, \texttt{FYI}, \texttt{STS}, \texttt{ERR}). When a message string is composed the first three letters of it are one of the above. Hence the receiving party can recognize the type of message being sent and act accordingly :
\begin{itemize}
\item \texttt{MYM} (Make your move) - messages the client-host asks the client to make their move
\item \texttt{FYI} (For your information) - usually send the tic tac toe board
\item \texttt{STS} (Status) - send the stage/status of the game i.e \texttt{Win of player}, \texttt{Draw}, \texttt{Continue}
\item \texttt{ERR} (Error) - There is an invalid move happening.
\end{itemize} 
As mentioned above the \texttt{FYI} send the tic tac toe board which is an \texttt{Array} of size 9. 
\subsubsection{Client (host)}
The client host (server as referred in the code) does most of the computation of the game.  The "server" recursively does the following in the \texttt{server\_main}
\begin{enumerate}
\item Waits for himself to make a valid move (a move inside the board that was not previously filled).
\item Updates the board with the current move
\item Checks if there is a winner or draw; If yes then it sends the status to the client and game ends i.e. stops the connection.
\item Otherwise sends the board to the client and waits for a response.
\item When the client sends a move the host checks if the move is valid. If not then repeat step 4. 
\item Otherwise repeat step 2 and 3. If game has not ended then go back to step 1.
\end{enumerate}
\subsubsection{Client }
The client has a simplier process. It only receives messages from the host and acts accordingly to the type of message received:
\begin{enumerate}
\item \texttt{MYM} message; it makes a move by clicking on the board; the move is converted into an index of the array and then send it to the host.
\item \texttt{FYI} gets the board and displays it
\item \texttt{ERR} the client made a wrong move
\item \texttt{STS} receives the status of the game. 
\end{enumerate}
All the following can be found in the \texttt{client\_main} function.
\subsection{Usage}
\subsubsection{Run}
To play the game Tic Tac Toe run:\\\\
\text{\texttt{make tictactoe}}\\
\text{\texttt{./tictactoe}}\\\\
Upon execution there is a graphic window appearing. The default measurements for the window are 1000 by 620 and the board is 600 by 600. The players could insert a move by simply clicking the position they like. The first player to play is \texttt{X} and the second \texttt{O}. If you want to customize the measurements of the window you could run the following code:\\\\
\texttt{./tictactoe -length\_window x -width\_window x -length x -width x}
  








\end{document}
